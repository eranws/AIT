\documentclass{article}
\usepackage{amssymb}
\usepackage{epsf}

\newtheorem{theorem}{Theorem}
\newtheorem{definition}{Definition}
\newtheorem{corollary}{Corollary}
\newtheorem{claim}{Claim}
\newcommand{\tup}[1]{\langle #1 \rangle}
\newcommand{\pref}[2]{(#1:#2)}
\newcommand{\idx}[2]{#1[#2]}
%\newcommand{\scott}[1]{[#1]}
\newcommand{\bnd}[2]{#1^{#2[]}}
\newcommand{\brr}[1]{\overline{#1}}
\newcommand{\baz}[1]{\lambda^0 #1.\ }
\newcommand{\basz}[1]{\lambda^1 #1.\ }
\newcommand{\bassz}[1]{\lambda^2 #1.\ }
\newcommand{\basssz}[1]{\lambda^3 #1.\ }
\newcommand{\Ct}{{\bf true}}
\newcommand{\Cf}{{\bf false}}
\newcommand{\CS}{{\bf S}}
\newcommand{\CK}{{\bf K}}
\newcommand{\CZ}{{\bf 0}}
\newcommand{\CO}{{\bf 1}}
\newcommand{\CP}{{\bf P}}
\newcommand{\CX}{{\bf X}}
\newcommand{\CY}{{\bf Y}}
\newcommand{\CI}{{\bf I}}
\newcommand{\CU}{{\bf U}}
\newcommand{\CE}{{\bf E}}
\newcommand{\CF}{{\bf F}}
\newcommand{\COm}{{\bf \Omega}}
\newcommand{\Cnil}{{\bf nil}}
\newcommand{\Csucc}{{\bf succ}}
\newcommand{\Cnull}{{\bf null}}
\newenvironment{proof}{\noindent{\bf Proof: }}

%\title{Binary Lambda Calculus and Combinatory Logic}
%\title{Binary Lambda Calculus and Algorithmic Information Theory}
\title{Functional Bits: Lambda Calculus based Algorithmic Information Theory }
\author{John Tromp}

\begin{document}

\maketitle

\begin{abstract}
%This paper aims to provide the simplest possible concrete definition
%of descriptional complexity---the length in bits of a shortest description
In the first part we introduce binary representations of both lambda calculus
and combinatory logic, together with very concise interpreters that witness
their simplicity. Along the way we present a simple graphical notation
for lambda calculus, a new empty list representation, improved bracket
abstraction, and a new fixpoint combinator.
In the second part we review Algorithmic Information Theory, for which
these interpreters provide a convenient vehicle.
We demonstrate this with several concrete upper bounds
on program-size complexity.
%including an elegant self-delimiting code for binary strings.
\end{abstract}

\section{Introduction}

The ability to represent programs as data and to map such data back
to programs (known as reification and reflection~\cite{fw}),
is of both practical use in
metaprogramming~\cite{dm} as well as theoretical use in computability and
logic~\cite{geb}.
It comes as no surprise that the pure lambda calculus,
which represents both programs and data as functions,
is well equipped to offer these features.
In~\cite{kl}, Kleene was the first to propose an encoding of lambda terms,
mapping them to G\"{o}del numbers, which can in turn be represented as
so called Church numerals.
Decoding such numbers is somewhat cumbersome, and not particularly efficient.
In search of simpler constructions, various alternative encodings
have been proposed using higher-order abstract syntax~\cite{PE88}
combined with the standard lambda representation of signatures \cite{SM89}.
A particularly simple encoding was proposed by Mogensen~\cite{m00},
for which the term $\lambda m.m(\lambda x.x)(\lambda x.x)$ acts as a
selfinterpreter.
The prevalent data format, both in information theory and in practice,
however, is not numbers, or syntax trees, but bits.
We propose binary encodings of both lambda and combinatory
logic terms, and exhibit relatively simple
and efficient interpreters (using the standard representation of
bit-streams as lists of booleans).

This gives us a representation-neutral notion of the size of a term,
measured in bits.
More importantly, it provides a way to describe arbitrary
data with, in a sense, the least number of bits possible.
We review the notion of how a computer reading bits and outputting
some result constitutes a description method, and how universal computers
correspond to optimal description methods.
We then pick specific universal computers based on our interpreters
and prove several of the basic results of Algorithmic Information Theory
with explicit constants.

\section{Lambda Calculus}
We only summarize the basics here. For a comprehensive treatment
we refer the reader to the standard reference~\cite{b}.

Assume a countably infinite set of {\em variables}
\[a,b,\ldots,x,y,z,x_0,x_1,\ldots\]
The set of lambda terms $\Lambda$ is built up from variables
using {\em abstraction} \[ (\lambda x.M) \]
and {\em application} \[ (M\ N), \]
where $x$ is any variable and $M,N$ are lambda terms.
$(\lambda x.M)$ is the function that maps $x$ to $M$,
while $(M\ N)$ is the application of function $M$ to argument $N$.
We sometimes omit parentheses, understanding abstraction to associate to
the right, and application to associate to the left, e.g.
$\lambda x.\lambda y.x\ y\ x$ denotes $(\lambda x.(\lambda y.((x\ y)x)))$.
We also join consecutive abstractions as in $\lambda x\ y.x\ y\ x$.

The free variables $FV(M)$ of a term $M$ are those variables not bound
by an enclosing abstraction. $\Lambda^0$ denotes the set of closed terms,
i.e. with no free variables. The simplest closed term is the identity
$\lambda x.x$.

We consider two terms {\em identical} if they only differ in the names
of bound variables, and denote this with $\equiv$, e.g.
$\lambda y.y\ x \equiv \lambda z.z\ x$.
The essence of $\lambda$ calculus is embodied in
the $\beta$-conversion rule which equates
\[(\lambda x.M)N = M[x:=N],\] where $M[x:=N]$ denotes the result
of substituting $N$ for all free occurrences of $x$ in $M$ (taking care
to avoid variable capture by renaming bound variables in $M$ if necessary).
For example,
\[(\lambda x\ y.y\ x)y \equiv (\lambda x.(\lambda z.z\ x))y
\equiv (\lambda x\ z.z\ x)y = \lambda z.z\ y.\]
A term with no $\beta$-redex, that is, no subterm of the form $(\lambda x.M)N$,
is said to be in {\em normal form}. Terms may be viewed as
denoting computations of which $\beta$-reductions form the steps,
and which may halt with a normal form as the end result.

\subsection{Some useful lambda terms}
Define (for any $M,P,Q,\ldots,R$)
\begin{eqnarray*}
\CI & \equiv & \lambda x.x \\
\Ct & \equiv & \lambda x\ y. x \\
\Cnil \equiv \Cf & \equiv & \lambda x\ y. y \\
%\tup{P} & \equiv & \lambda z.z\ P \\
%\tup{P,Q} & \equiv & \lambda z.z\ P\ Q \\
\tup{P,Q,\ldots,R} & \equiv & \lambda z.z\ P\ Q \ldots\ R \\
%\Cnil & \equiv & \lambda x.\Ct \\
%\Cnil & \equiv & \Cf \\
% \Csucc & \equiv & \lambda n\ z\ s.\ s\ n \\
%\scott{0} & \equiv & \Cnil \\
\idx{M}{0} & \equiv & M\ \Ct \\
\idx{M}{i+1} & \equiv & \idx{(M\ \Cf)}{i} \\
%\Cnull& \equiv & \tup{\lambda a\ b\ c.\ \Cf, \Ct} \\
\CY & \equiv & \lambda f.((\lambda x.f\ (x\ x))(\lambda x.f\ (x\ x))) \\
\COm & \equiv & (\lambda x.x\ x)(\lambda x.x\ x)
\end{eqnarray*}

Note that
\[ \Ct\ P\ Q = (\lambda x\ y.x)\ P\ Q = x[x:=P] = P \]
\[ \Cf\ P\ Q = (\lambda x\ y.y)\ P\ Q = y[y:=Q] = Q, \]
justifying the use of these terms as representing the booleans.

A pair of terms like $P$ and $Q$ is represented by $\tup{P,Q}$, which
allows one to retrieve its parts by applying $\tup{\Ct}$ or $\tup{\Cf}$:
\[ \tup{\Ct} \tup{P,Q} = \tup{P,Q}\ \Ct = \Ct\ P\ Q = P\]
\[ \tup{\Cf} \tup{P,Q} = \tup{P,Q}\ \Cf = \Cf\ P\ Q = Q.\]

Repeated pairing is the standard way of representing a sequence
of terms: \[\tup{P,\tup{Q,\tup{R,\ldots}}}.\]
A sequence is thus represented by pairing its first element
with its {\em tail}---the sequence of remaining elements.
The $i$'th element of a sequence $M$ may be selected as
$\idx{M}{i}$. To wit:
\[ \idx{\tup{P,Q}}{0} = \Ct\ P\ Q = P, \]
\[ \idx{\tup{P,Q}}{i+1} \equiv \idx{(\tup{P,Q}\ \Cf)}{i} = \idx{Q}{i}. \]

The empty sequence, for lack of a first element, cannot be
represented by any pairing, and is instead represented by $\Cnil$.
A finite sequence $P,Q,\ldots, R$ can thus be represented as
$\tup{P,\tup{Q,\tup{\ldots,\tup{R,\Cnil}\ldots}}}$.
%which equals $[P,Q,\ldots,R]\Cnil$. We call $[P,Q,\ldots,R]$
%a {\em prefix} since it prefixes the elements
%$P,Q,\ldots,R$ before a given sequence.
%Note that $\tup{} \equiv [] \equiv \CI$.

Our choice of $\Cnil$ allows for the processing
of a possible empty list $s$ with the expression
\[ s\ M\  N, \]
which for $s\equiv \Cnil$ reduces to $N$, and for
$s\equiv \tup{P,Q}$ reduces to $M\ P\ Q\ N$.
%(normally $c$ does not occur free in $M$, but it can be convenient at times).
%This is simpler than the expression
%\[ \Cnull\ s\ N (s\ (\lambda a\ b.\ M)), \]
%which achieves the same effect with the help of the null-test function $\Cnull$.
%(also note that $s\ (\lambda a\ b\ c.\ \Ct)\ s$ tests whether $s$ is non-null.)
In contrast, Barendregt~\cite{baay} chose $\CI$ to represent the empty list,
which requires a more complicated list processing expression like
like $s\ (\lambda a\ b\ c.c\ a\ b)\ M X N$,
which for $s= \Cnil$ reduces to $N\ M\ X$, and for
$s\equiv \tup{P,Q}$ reduces to $M\ P\ Q\ X\ N$.

$\CY$ is the {\em fixpoint} operator, that
satisfies \[\CY f = (\lambda x.f\ (x\ x))(\lambda x.f\ (x\ x)) = f\ (\CY\ f).\]
This allows one to transform a recursive definition $f= \ldots f \ldots$
into $f = \CY(\lambda f. (\ldots f \ldots))$, which behaves exactly as desired.

$\COm$ is the prime example of a term with no normal form, the equivalence
of an infinite loop.

\subsection{Binary strings}
Binary strings are naturally represented by boolean sequences,
where $\Ct$ represents $0$ and $\Cf$ represents $1$.
\begin{definition}
For a binary string $s$ and lambda term $M$, $\pref{s}{M}$ denotes the
list of booleans corresponding to $s$, terminated with $M$.
Thus, $\pref{s}{\Cnil}$ is the standard representation of string $s$.
\end{definition}
For example, $\pref{011}{\Cnil} \equiv \tup{\Ct,\tup{\Cf,\tup{\Cf,\Cnil}}}$
represents the string $011$.
We represent an unterminated string, such as part of an input stream,
as an open term $\pref{s}{z}$, where the free variable $z$ represents
the remainder of input.

\subsection{de Bruijn notation}
In~\cite{bruijn}, de Bruijn proposed an alternative notation for closed
lambda terms using natural numbers rather than variable names.
Abstraction is simply written
$\lambda M$ while the variable bound by the $n$'th enclosing
$\lambda$ is written as the index $n$. In this notation,
$\lambda x\ y\ z.z\ x\ y \equiv \lambda\ \lambda\ \lambda\ 0\ 2\ 1$.
It thus provides a canonical notation for all identical terms.
Beta-conversion in this notation avoids variable capture,
but instead requires {\em shifting} the indices, i.e. adjusting them
to account for changes in the lambda nesting structure.
Since variable/index exchanges don't affect each other,
it's possible to mix both forms of notation, as we'll do later.

\subsection{graphical notation}
{\em Lambda Diagrams} are a graphical notation for closed lambda terms,
in which abstractions are represented by horizontal lines, variables by vertical lines emanating down from their binding lambda, and applications by horizontal links connecting the leftmost variables. In the alternative style, applications link the nearest deepest variables, for a more stylistic, if less uniform, look
\subsection{Binary Lambda Calculus}

\begin{definition}
The code for a term in de Bruijn notation is defined
inductively as follows:
\begin{eqnarray*}
\widehat{n} & \equiv & 1^{n+1} 0 \\
\widehat{\lambda M} & \equiv & 0 0 \widehat{M} \\
\widehat{M N} & \equiv & 0 1 \widehat{M}\ \widehat{N} \\
\end{eqnarray*}
We call $|\widehat{M}|$ the {\em size} of $M$.
\end{definition}

For example
$\widehat{\CI} \equiv 0010$,
$\widehat{\Cf} \equiv 000010$,
$\widehat{\Ct} \equiv 0000110$ and
$\widehat{\lambda x.x\ x} \equiv 00011010$,
$\widehat{\lambda x.\Cf} \equiv 00000010$,
of sizes 4,6,7,8 and 8 bits respectively,
are the 5 smallest closed terms.

The main result of this paper is the following
\begin{theorem}
There is a self-interpreter $\CE$ of size 206
such that for every closed term $M$ and terms $C,N$ we have
\[ \CE\ C\ \pref{\widehat{M}}{N} = C\ (\lambda z.M)\ N \]
\label{int206}
\end{theorem}
The interpreter works in continuation passing style~\cite{fwh}.
Given a continuation and a bitstream containing an encoded term,
it returns the continuation applied to the abstracted decoded term 
and the remainder of the stream. The reason for the abstraction
becomes evident in the proof.

The theorem is a special case of a stronger
one that applies to arbitrary de Bruijn terms.
Consider a de Bruijn term $M$
in which an index $n$ occurs at a depth of
$i \leq n$ nested lambda's. E.g., in $M \equiv \lambda 3$,
the index $3$ occurs at depth $1$. This index is like
a free variable in that it is not bound within $M$.
The interpreter (being a closed term) applied to other closed terms,
cannot produce anything but a closed term. So it cannot possibly
reproduce $M$. Instead, it produces
terms that expect a list of bindings for free indices.
These take the form $\bnd{M}{z}$, which is defined
as the result of replacing every free index in $M$, say $n$
at depth $i\leq n$, by $\idx{z}{n-i}$. 
For example, $\bnd{(\lambda 3)}{z} = \lambda \idx{z}{3-1} =
\lambda (z\ \Cf\ \Cf\ \Ct)$,
selecting binding number $2$ from binding list $z$.

The following claim (using mixed notation) will be needed later.
\begin{claim}
For any de Bruijn term $M$, we have
$\bnd{(\lambda M)}{z} = \lambda y.\bnd{M}{\tup{y,z}}$
\label{lift}
\end{claim}

\begin{proof}
A free index $n$ at depth $i \leq n$ in $M$,
gets replaced by $\idx{\tup{y,z}}{n-i}$ on the right.
If $i<n$ then $n$ is also free in $\lambda M$ at
depth $i+1$ and gets replaced by $\idx{z}{n-i-1} = \idx{\tup{y,z}}{n-i}$.
If $i=n$ then $n$ is bound by the front $\lambda$, while
$\idx{\tup{y,z}}{n-i} = \idx{\tup{y,z}}{0} = y$.
\end{proof}

To prove Theorem~\ref{int206} it suffices to prove the more general:
\begin{theorem}
There is a self-interpreter $\CE$ of size 206,
such that for all terms $M,C,N$ we have
\[ \CE\ C\ \pref{\widehat{M}}{N} = C\ (\lambda z.\bnd{M}{z})\ N \]
\label{gen210}
\end{theorem}

\begin{proof}
We take
\begin{eqnarray*}
\CE & \equiv & \CY\ (\lambda e\ c\ s.s\ (\lambda a\ t.t\ (\lambda b.a\ \CE_0\ \CE_1))) \\
\CE_0 & \equiv & e\ (\lambda x.b\ (c\ (\lambda z\ y.x\ \tup{y,z}))
         (e\ (\lambda y.c\ (\lambda z.x\ z\ (y\ z))))) \\
\CE_1 & \equiv & 
(b\ (c\ (\lambda z.z\ b))(\lambda s.e\ (\lambda x.c\ (\lambda z.x\ (z\ b)))\ t))
\end{eqnarray*}
of size 217 and note that the beta reduction from
$\CY\ M$ to $(\lambda x.x\ x)(\lambda x.M\ (x\ x))$
saves 7 bits, while\cite{bertram} replacing $\CE$ by
$\CY\ (\lambda e\ c\ s.s\ (\lambda a\ t\ e\ c.t\ (\lambda b.a\ \CE_0\ \CE_1))e c)$
and doing one more beta reduction of $(\lambda x.M\ (x\ x))$,
saves another 4 bits.

Recall from the discussion of $\CY$ that the above is a transformed
recursive definition where $e$ will take the value of $\CE$.

Intuitively, $\CE$ works as follows. Given a continuation $c$ and
sequence $s$, it extracts the leading bit $a$ and tail $t$ of $s$,
extracts the next bit $b$, and selects $\CE_0$ to deal with $a=\Ct$
(abstraction or application), or $\CE_1$ to deal with $a=\Cf$ (an index).

$\CE_0$ calls $\CE$ recursively, extracting a decoded term $x$.
In case $b=\Ct$ (abstraction), it prepends a new variable $y$ to
bindings list $z$, and returns the continuation applied to the decoded
term provided with the new bindings.
In case $b=\Cf$ (application), it calls $\CE$ recursively again,
extracting another decoded term $y$, and returns the continuation
applied to the application of the decoded terms provided with shared bindings.

$\CE_1$, in case $b=\Ct$, decodes to the $0$ binding selector.
In case $b=\Cf$, it calls $\CE$ recursively on $t$ (coding for an
index one less) to extract a binding selector $x$, which is provided with
the tail $z\ b$ of the binding list to obtain the correct selector.

We continue with the formal proof, using induction on $M$.

Consider first the case where $M=0$. Then
\begin{eqnarray*}
\CE\ C\ \pref{\widehat{M}}{N} & = & \CE\ C\ \pref{10}{N} \\
& = & \tup{\Cf,\tup{\Ct,N}}\ (\lambda a\ t.t\ (\lambda b.a\ \CE_0\ \CE_1)) \\
& = & \tup{\Ct,N}\ (\lambda b.\Cf\ \CE_0\ \CE_1) \\
& = & (\CE_1\ N)[b:=\Ct] \\
& = & C\ (\lambda z.z\ \Ct)\ N,
\end{eqnarray*}
as required.
Next consider the case where $M=n+1$. Then, by induction,
\begin{eqnarray*}
 \CE\ C\ \pref{\widehat{M}}{N} & = & \CE\ C\ \pref{1^{n+2}0}{N} \\
& = & \tup{\Cf,\tup{\Cf,\pref{1^n0}{N}}}\ (\lambda a\ t.t\ (\lambda b.a\ \CE_0\ \CE_1)) \\
& = & (\lambda s.e\ (\lambda x.C\ (\lambda z.x\ (z\ \Cf))) \pref{1^{n+1}0}{N})\pref{1^n0}{N} \\
& = & \CE\ (\lambda x.C\ (\lambda z.x\ (z\ \Cf)))\ \pref{\widehat{n}}{N} \\
& = & (\lambda x.C\ (\lambda z.x\ (z\ \Cf)))\ (\lambda z.\bnd{n}{z})\ N \\
& = & C\ (\lambda z.\bnd{n}{(z\ \Cf)})\ N \\
& = & C\ (\lambda z.\idx{(z\ \Cf)}{n}))\ N \\
& = & C\ (\lambda z.\idx{z}{n+1}))\ N \\
& = & C\ (\lambda z.\bnd{(n+1)}{z})\ N,
\end{eqnarray*}
as required.
Next consider the case $M=\lambda M'$. Then, by induction and claim~\ref{lift},
\begin{eqnarray*}
\CE\ C\ (\pref{\widehat{\lambda M'}}{N}) & = & \CE\ C\ \pref{00\widehat{M'}}{N} \\
& = & \tup{\Ct,\tup{\Ct,\pref{\widehat{M'}}{N}}}\ (\lambda a\ t.t\ (\lambda b.a\ \CE_0\ \CE_1)) \\
& = & e\ (\lambda x.(C\ (\lambda z\ y.x\tup{y,z})))\ \pref{\widehat{M'}}{N} \\
& = & (\lambda x.(C\ (\lambda z\ y.x\ \tup{y,z}))) (\lambda z.\bnd{M'}{z})\ N \\
& = & C\ (\lambda z\ y.(\lambda z.\bnd{M'}{z})\ \tup{y,z})\ N \\
& = & C\ (\lambda z. (\lambda y.\bnd{M'}{\tup{y,z}}))\ N \\
& = & C\ (\lambda z. \bnd{(\lambda M')}{z}))\ N,
\end{eqnarray*}
as required.
Finally consider the case $M=M'\ M''$. Then, by induction,
\begin{eqnarray*}
\CE\ C\ \pref{\widehat{M'\ M''}}{N} & = & \CE\ C\ \pref{01\widehat{M'}\ \widehat{M''}}{N} \\
& = & \tup{\Ct,\tup{\Cf,\pref{\widehat{M'}\ \widehat{M''}}{N}}}(\lambda a\ t.t\ (\lambda b.a\ \CE_0\ \CE_1)) \\
& = & e\ (\lambda x.(e\ (\lambda y.C\ (\lambda z.x\ z\ (y\ z)))))\ \pref{\widehat{M'}\ \widehat{M''}}{N} \\
& = & (\lambda x.(e\ (\lambda y.C\ (\lambda z.x\ z\ (y\ z))))) (\lambda z.\bnd{M'}{z})\ \pref{\widehat{M''}}{N} \\
& = & e\ (\lambda y.C\ (\lambda z.(\lambda z.\bnd{M'}{z})\ z\ (y\ z))) \pref{\widehat{M''}}{N} \\
& = & (\lambda y.C\ (\lambda z.\bnd{M'}{z}\ (y\ z))) (\lambda z.\bnd{M''}{z})\ N \\
& = & C\ (\lambda z.\bnd{M'}{z}\ \bnd{M''}{z})\ N \\
& = & C\ (\lambda z.\bnd{(M'\ M'')}{z})\ N,
\end{eqnarray*}
as required. This completes the proof of Theorem~\ref{int206}.
\end{proof}

% 206: (\1 1) (\\\1   (\\\\3 (\5 (3 (\2 (3 (\\3 (\1 2 3))) (4 (\4 (\3 1 (2 1)))))) (1 (2 (\1 2)) (\4 (\4 (\2 (1 4))) 5))))  (3 3) 2)
% 210: (\1 1) (\(\\\1 (\\  1 (\3 (6 (\2 (6 (\\3 (\1 2 3))) (7 (\7 (\3 1 (2 1)))))) (1 (5 (\1 2)) (\7 (\7 (\2 (1 4))) 3))))) (1 1))

We conjecture that any self-interpreter for any binary representation of lambda calculus
must be at least 24 bytes in size, which would make $E$ close to optimal.

\section{Combinatory Logic}

Combinatory Logic (CL) is the equational theory of {\em combinators}---terms
built up, using application only, from the two constants $\CK$ and $\CS$,
which satisfy
\begin{eqnarray*}
\CS\ M\ N\ L & = & M\ L\ (N\ L) \\
\CK\ M\ N & = & M
\end{eqnarray*}
CL may be viewed as a subset of lambda calculus, in which
$\CK\equiv \lambda x\ y.x$,
$\CS\equiv \lambda x\ y\ z.x\ z\ (y\ z)$,
and where the beta conversion rule can only be applied groupwise,
either for an $\CS$ with 3 arguments, or for a $\CK$ with 2 arguments.
Still, the theories are largely the same, becoming equivalent in the
presence of the rule of extensionality (which says $M=M'$ if $M\ N= M'\ N$
for all terms $N$).

A process known as {\em bracket abstraction} allows for the translation
of any lambda term to a {\em combination}---a CL term
containing variables in addition to $\CK$ and $\CS$.
It is based on the following identities, which are easily verified:
\begin{eqnarray*}
\lambda x.x = \CI & = & \CS\ \CK\ \CK \\
\lambda x.M & = & \CK\ M \,\,\,\,\,\,\,\,\,\,(x \mbox{ not free in } M) \\
\lambda x.M\ N & = & \CS\ (\lambda x.M)\ (\lambda x.N)
\end{eqnarray*}

$\lambda$'s can thus be successively eliminated, e.g.:
\begin{eqnarray*}
\lambda x\ y.y\ x & \equiv & \lambda x\ (\lambda y.y\ x) \\
& = & \lambda x\ (\CS\ \CI (\CK\ x)) \\
& = & \CS\ (\CK\ (\CS\ \CI)) (\CS\ (\CK\ \CK)\ \CI),
\end{eqnarray*}
where $\CI$ is considered a shorthand for $\CS\ \CK\ \CK$.

Bracket abstraction is an operation $\lambda^0$ on combinations $M$
with respect to a variable $x$, such that the resulting combination
contains no occurrence of $x$ and behaves as $\lambda x.M$:
\begin{eqnarray*}
\baz{x}x & \equiv & \CI \\
\baz{x}M & \equiv & \CK\ M \,\,\,\,\,\,\,\,(x \not\in M) \\
\baz{x}(M\ N) & \equiv & \CS\ (\baz{x}M)\ (\baz{x}N)
\end{eqnarray*}

\subsection{Binary Combinatory Logic}
Combinators have a wonderfully simple encoding as binary strings:
encode $\CS$ as $00$, $\CK$ as $01$, and application as $1$.

\begin{definition}
We define the encoding $\widetilde{C}$ of a combinator $C$ as
\begin{eqnarray*}
\widetilde{\CK} & \equiv & 00 \\
\widetilde{\CS} & \equiv & 01 \\
\widetilde{C\ D} & \equiv & 1\ \widetilde{C}\ \widetilde{D}
\end{eqnarray*}
Again we call $|\widetilde{C}|$ the {\em size} of combinator $C$.
\end{definition}

For instance, the combinator $\CS(\CK\CS\CS)\equiv(\CS((\CK\CS)\CS))$
is encoded as $1\ 01\ 1\ 1\ 00\ 01\ 01$.
The size of a combinator with $n$ $\CK/\CS$'s, which necessarily
has $n-1$ applications, is thus $2n+n-1=3n-1$.

For such a simple language we expect a similarly simple interpreter.
\begin{theorem}
There is a cross-interpreter $\CF$ of size 119,
such that for every combinator $M$ and terms $C,N$ we have
\[ \CF\ C\ \pref{\widetilde{M}}{N} = C\ M\ N \]
\label{int119}
\end{theorem}

\begin{proof}
We take
\begin{eqnarray*}
\CF & \equiv & \CY\ (\lambda e\ c\ s.s(\lambda a.a\ \CF_0\ \CF_1)) \\
\CF_0 & \equiv & \lambda t.t\ (\lambda b.c\ (b\ \CK\ \CS)) \\
\CF_1 & \equiv & e\ (\lambda x.e\ (\lambda y.(c\ (x\ y))))
\end{eqnarray*}
of size 131 and note that a toplevel beta reduction saves 7 bits in size,
while replacing $\CK$ by $b$ saves another 5 bits (we don't define $\CF$ that way
because of its negative impact on bracket abstraction).

Given a continuation $c$ and
sequence $s$, it extracts the leading bit $a$ of $s$,
and tail $t$ extracts the next bit $b$,
and selects $\CF_0$ to deal with $a=\Ct$
($\CK$ or $\CS$), or $\CF_1$ to deal with $a=\Cf$ (application).
Verification is straightforward and left as an exercise to the reader.
\end{proof}

We conjecture that any self-interpreter for any binary representation of combinatory logic must be at least 14 bytes in size.
The next section considers translations of $F$ which yield
a self-interpreter of CL.

\subsection{Improved bracket abstraction}

The basic form of bracket abstraction is not particularly
efficient. Applied to $\CF$, it produces a combinator of size 536.

A better version is $\lambda^1$, which uses the additional rule
\[ \basz{x}(M\ x) \equiv M \,\,\,\,\,\,\,\,(x \not\in M) \]
whenever possible.
Now the size of $\CF$ as a combinator is only 281, just over half as big.

Turner~\cite{t} noticed that repeated use of bracket abstraction
can lead to a quadratic expansion on terms such as
\[ \CX \equiv \lambda a\ b\ \ldots\ z.(a\ b\ \ldots\ z)\ (a\ b\ \ldots\ z), \]
and proposed new combinators to avoid such behaviour.
We propose to achieve a similar effect with the following
set of 9 rules in decreasing order of applicability:
%of which only the first is to take precedence
%over rule $\baz{x}M \equiv \CK\ M$:
\begin{eqnarray*}
\bassz{x}(\CS\ \CK\ M) & \equiv & \CS\ \CK\,\,\,\,(\mbox{for all $M$}) \\
\bassz{x}M & \equiv & \CK\ M \,\,\,\,\,\,\,\,(x \not\in M) \\
\bassz{x}x & \equiv & \CI \\
\bassz{x}(M\ x) & \equiv & M \,\,\,\,\,\,\,\,(x \not\in M) \\
\bassz{x}(x\ M\ x) & \equiv & \bassz{x}(\CS\ \CS\ \CK\ x\ M) \\
\bassz{x}(M\ (N\ L)) & \equiv & \bassz{x}(\CS\ (\bassz{x}M)\ N\ L)\,\,\,\,(\mbox{$M,N$ combinators}) \\
\bassz{x}((M\ N)\ L) & \equiv & \bassz{x}(\CS\ M\ (\bassz{x}L)\ N)\,\,\,\,(\mbox{$M,L$ combinators}) \\
\bassz{x}((M\ L)\ (N\ L)) & \equiv & \bassz{x}(\CS\ M\ N\ L)\,\,\,\,(\mbox{$M,N$ combinators}) \\
\bassz{x}(M\ N) & \equiv & \CS\ (\bassz{x}M)\ (\bassz{x}N)
\end{eqnarray*}
The first rule exploits the fact that $\CS\ \CK\ M$ behaves as identity,
whether $M$ equals $\CK,x$ or anything else.
The fifth rule avoids introduction of two $\CI s$.
The sixth rule prevents occurrences of $x$ in $L$ from becoming too
deeply nested, while the seventh does the same for occurrences of $x$ in $N$.
The eighth rule abstracts an entire expression $L$ to avoid duplication.
The operation $\bassz{x}M$ for combinators $M$
will normally evaluate to $\CK\ M$,
but takes advantage of the first rule by considering any $\CS\ \CK\ M$
a combinator.
Where $\lambda^1$ gives an $\CX$ combinator of size 2030, $\lambda^2$ brings
this down to 374 bits.

For $\CF$ the improvement is more modest, to 275 bits.
For further improvements we turn our attention to the unavoidable
fixpoint operator.

$\CY$, due to Curry, is of minimal size in the $\lambda$ calculus.
At 25 bits, it's 5 bits shorter than Turing's alternative fixpoint operator
\[ \CY' \equiv (\lambda z.z\ z)(\lambda z.\lambda f.f\ (z\ z\ f)). \]
But these translate to combinators of size 65 and 59 bits respectively.

In comparison, the fixpoint operator
\[ \CY'' \equiv (\lambda x\ y.x\ y\ x)(\lambda y\ x. y(x\ y\ x)) \]
translates to combinator \[\CS\ \CS\ \CK\ (\CS\ (\CK\ (\CS\ \CS\ (\CS\ (\CS\ \CS\ \CK))))\ \CK)\]
of size 35, the smallest possible fixpoint combinator as verified by
exhaustive search by computer.

(The situation is similar for $\COm$ which yields a combinator of size $41$,
while $\CS\ \CS\ \CK\ (\CS\ (\CS\ \CS\ \CK))$, of size $20$, is the smallest
{\em unsolvable} combinator---the equivalent of an undefined result,
see~\cite{b}).

Using $\CY''$ instead of $\CY$ gives us the following
\begin{theorem}
There is a self-interpreter $\CF$ for Combinatory Logic of size 263.
\label{int263}
\end{theorem}

Comparing theorems~\ref{int119} and \ref{int263}, we conclude that
$\lambda$-calculus is a much more concise language than CL. Whereas
in binary $\lambda$-calculus, an abstraction takes only 2 bits plus
$i+1$ bits for every occurrence of the variable at depth $i$, in binary
CL the corresponding bracket abstraction typically introduces at least one,
and often several $\CS$'s and $\CK$'s (2 bits each)
per level of depth per variable occurrence.

\section{Program Size Complexity}

Intuitively, the amount of information in an object
is the size of the shortest program that outputs the object.
The first billion digits of $\pi$ for example, contain little information,
since they can be calculated by a program of a few lines only.
Although information content may seem to be highly dependent on
choice of programming language,
the notion is actually invariant up to an additive constant.

The theory of program size complexity, which has become known as
{\em Algorithmic Information Theory} or {\em Kolmogorov complexity}
after one of its founding fathers,
has found fruitful application in many fields such as combinatorics,
algorithm analysis, machine learning, machine models, and logic.

In this section we propose a concrete definition of Kolmogorov complexity
that is (arguably) as simple as possible, by turning the above interpreters
into a `universal computer'.

Intuitively, a computer is any device that can read bits from an input stream,
perform computations, and (possibly) output a result.
Thus, a computer is a method of description in the sense that the string
of bits read from the input describes the result.
A universal computer
is one that can emulate the behaviour of any other computer when
provided with its description.
Our objective is to define, concretely, for any object $x$,
a measure of complexity of description $C(x)$ that shall be the length
of its shortest description. This requires
fixing a description method, i.e. a computer.
By choosing a universal computer, we achieve invariance: the complexity
of objects is at most a constant greater than under any other description
method.

Various types of computers have been considered in the past as
description methods.

Turing machines are an obvious choice, but turn out to be less than ideal:
The operating logic of a Turing machine---its {\em finite control}---is
of an irregular nature, having no straightforward encoding into a bitstring.
This makes construction of a universal Turing machine that has to parse and
interpret a finite control description quite challenging.
Roger Penrose takes up this challenge in his book~\cite{pen},
at the end of Chapter~2, resulting in a 
universal Turing machine whose own encoding is an impressive 5495 bits in size,
over 26 times that of $\CE$.

The ominously named language `Brainfuck' which advertises itself as
``An Eight-Instruction Turing-Complete Programming Language''~\cite{bf},
can be considered a streamlined form of Turing machine. Indeed,
Oleg Mazonka and Daniel B. Cristofani~\cite{mc} managed to write a very clever
BF self-interpreter of only 423 instructions, which translates to
$423 \cdot \log(8)=1269$ bits
(the alphabet used is actually ASCII at 7 or 8 bits per symbol,
but the interpreter could be redesigned to use 3-bit symbols
and an alternative program delimiter).

In \cite{lev}, Levin stresses the importance of a (descriptional complexity)
measure, which, when
compared with other natural measures, yields small
constants, of at most a few hundred bits. His approach is based on
{\em constructive objects} (c.o.'s) which are functions from and to
lower ranked c.o.'s. Levin stops short of exhibiting a specific
universal computer though, and the abstract, almost topological,
nature of algorithms in the model complicates a study of the constants
achievable.

In~\cite{ch}, Gregory Chaitin paraphrases John McCarthy about his
invention of LISP,
as ``This is a better universal Turing machine.
Let's do recursive function theory that way!''
Later, Chaitin continues with ``So I've done that using LISP because LISP is
simple enough, LISP is in the intersection between
theoretical and practical programming.
Lambda calculus is even simpler and more elegant than LISP,
but it's unusable. Pure lambda calculus with combinators S and K,
it's beautifully elegant, but you
can't really run programs that way, they're too slow.''

There is however nothing intrinsic to $\lambda$ calculus or CL that is slow;
only such choices as Church numerals for arithmetic
can be said to be slow,
but one is free to do arithmetic in binary rather than in unary.
Frandsen and Sturtivant~\cite{fs} amply demonstrate the efficiency
of $\lambda$ calculus with a linear time implementation of $k$-tree
Turing Machines.
Clear semantics should be a primary concern, and Lisp is somewhat
lacking in this regard~\cite{mul}.
This paper thus develops the approach suggested but
discarded by Chaitin.

\subsection{Functional Complexity}

By providing the appropriate continuations to the
interpreters that we constructed, they become
universal computers describing functional terms modulo equality.
%or equivalently, describing normal forms.
Indeed, for
\begin{eqnarray*}
\CU & \equiv & \CE\ \tup{\COm} \\
\CU' & \equiv & \CF\ \CI
\end{eqnarray*}
of sizes $|\widehat{\CU}|=232$ and $|\widetilde{\CU'}|=272$,
Theorems~\ref{int206} and \ref{int119} give
\begin{eqnarray*}
\CU\ \pref{\widehat{M}}{N} & = & M\ N \\
\CU'\ \pref{\widetilde{M}}{N} & = & M\ N
\end{eqnarray*}
for every closed $\lambda$-term or combinator $M$ and arbitrary $N$,
immediately establishing their universality.

The universal computers essentially define new binary languages,
which we may call {\em universal binary lambda calculus} and
{\em universal combinatory logic}, whose programs comprise two parts.
The first part is a program in one of the original binary languages,
while the second part is all the binary data that is consumed when
the first part is interpreted.
It is precisely this ability to embed
arbitrary binary data in a program that allows for universality.

Note that by Theorem~\ref{gen210},
the continuation $\tup{\COm}$ in $U$ results in a term
$\bnd{M}{\COm}$. For closed $M$, this term is identical to $M$,
but in case $M$ is not closed,
a free index $n$ at $\lambda$-depth $n$ is now bound to
$\idx{\COm}{n-n}$, meaning that any attempt to apply free indices diverges.
Thus the universal computer essentially forces programs to be closed terms.

% Alternatively, one could exploit free indices to extend the language.
% For example, a continuation $\tup{\tup{P,\tup{Q,\tup{R,\COm}}}}$
% would yield a universal machine $\CU^{P,Q,R}$ satisfying
% $\CU^{P,Q,R}\ \pref{\widehat{M}}{N} = (\lambda \lambda \lambda M)\ R\ Q\ P\ N$
% (assuming that $\lambda \lambda \lambda M$ is closed),
% thus providing access to functions $P,Q,R$
% without paying the price of $12+|\widehat{P}|+|\widehat{Q}|+|\widehat{R}|$.
% This might prove useful in certain specific applications, but for general theorems
% like we prove in this paper, the pure $\CU$ is to be preferred.

We can now define the Kolmogorov complexity of a term $x$,
which comes in three flavors.
In the {\em simple} version, programs are terminated with
$N=\Cnil$ and the result must equal $x$.
In the {\em prefix} version, programs are not terminated,
and the result must equal the pair of $x$ and the remainder of the input.
In both cases the complexity is conditional on zero or more terms $y_i$.

\begin{definition}
\begin{eqnarray*}
KS(x|y_1,\ldots,y_{k}) & = &
\min \{ l(p)\ |\ \CU\ \pref{p}{\Cnil}\ y_1\ \ldots\ y_{k} =      x    \} \\
KP(x|y_1,\ldots,y_{k}) & = &
\min \{ l(p)\ |\ \CU\ \pref{p}{ z   }\ y_1\ \ldots\ y_{k} = \tup{x,z} \}
\end{eqnarray*}
\label{defait}
\end{definition}

The definition also applies to infinite terms according to the {\em infinitary lambda calculus}
of \cite{KKSV97}.

In the special case of $k=0$ we obtain the unconditional complexities
$KS(x)$ and $KP(x)$.

Finally, for a binary string $s$, we can define its {\em monotone} complexity
as
\[ KM(s|y_1,\ldots,y_{k}) =
\min \{ l(p)\ |\ \exists M:
   \CU\ \pref{p}{\COm }\ y_1\ \ldots\ y_{k} = \pref{s}{M} \}.\] 
In this version, we consider the partial outputs produced by increasingly
longer prefixes of the input, and the complexity of $s$ is the shortest
program that causes the output to have prefix $s$.

\subsection{Monadic IO}
The reason for preserving the remainder of input in the prefix casse
is to facilitate the processing of concatenated descriptions,
in the style of monadic IO~\cite{spj}.
Although a pure functional language like $\lambda$ calculus cannot
define functions with side effects, as traditionally used to implement IO,
it can express an abstract data type representing IO actions; the IO monad.
In general, a monad consists of a type constructor and two functions,
{\em return} and {\em bind} (also written \verb% >>=% in infix notation)
which need to satisfy
certain axioms~\cite{spj}. IO actions can be seen as functions operating
on the whole state of the world, and returning a new state of the world.
Type restrictions ensure that IO actions can be combined only through
the bind function, which according to the axioms,
enforces a sequential composition in which the world is single-threaded.
Thus, the state of the world is never duplicated or lost.
In our case, the world of the universal machine consists of only the
input stream. The only IO primitive needed is {\bf readBit}, which maps
the world onto a pair of the bit read and the new world.
But a list is exactly that; a pair of the first element and the remainder.
So {\bf readBit} is simply the identity function!
The {\bf return} function, applied to some $x$, should map the
world onto the pair of $x$ and the unchanged world, so it is defined by
${\bf return} \equiv \lambda x\ y. \tup{x,y}$.
Finally, the bind function, given an action $x$ and a function $f$,
should subject the world $y$ to action $x$ (producing some $\tup{a,y'}$)
followed by action $f a$, which is defined by
${\bf bind} \equiv \lambda x\ f\ y. x\ y\ f$
(note that $\tup{a,y'}f=f\ a\ y'$)
One may readily verify that these definitions satisfy the monad axioms.
Thus, we can wite programs for $U$ either by processing the input stream
explicitly, or by writing the program in monadic style.
The latter can be done in the pure functional language `Haskell'~\cite{hk},
which is essentially typed lambda calculus with a lot of syntactic sugar.

\subsection{An Invariance Theorem}

The following theorem is the first concrete instance of the
Invariance Theorem, 2.1.1 in~\cite{lv}.

\begin{theorem}
Define $KS'(x|y_1,\ldots,y_{k})$ and $KP'(x|y_1,\ldots,y_{k})$
analogous to Definition~\ref{defait} in terms of $\CU'$.
Then $KS(x) \leq KS'(x)+125$ and $KP(x) \leq KP'(x)+125$.
\end{theorem}

The proof is immediate from
Theorem~\ref{int119} by using $\widehat{\CU'}$ of length 125 as
prefix to any program for $\CU'$.
We state without proof that
a redesigned $\CU$ translates to a combinator of size 617,
which thus forms an upper bound in the other direction.

Now that complexity is defined for as rich a class of objects as
terms (modulo equality), it is easy to extend it to other
classes of objects by mapping them into $\lambda$ terms.% in normal form.

For binary strings, this means mapping string $s$ onto the term
$\pref{s}{\Cnil}$. And for a tuple of binary strings $s_0,\ldots,s_k$,
we take $\tup{\pref{s_0}{\Cnil},\ldots,\pref{s_k}{\Cnil}}$.

We next look at numbers in more detail, revealing a link with
self-delimiting strings.

\subsection{Numbers and Strings}

Consider the following correspondence between
natural numbers, binary strings, and strings over $\{1,2\}$:
\[
\begin{array}{rccccccccccc}
n \in {\mathbb N}:    &        0 & 1 & 2 &  3 &  4 &  5 &  6 &   7 &   8 &   9 & \ldots \\
x \in \{0,1\}^{\ast}: & \epsilon & 0 & 1 & 00 & 01 & 10 & 11 & 000 & 001 & 010 & \ldots \\
y \in \{1,2\}^{\ast}: & \epsilon & 1 & 2 & 11 & 12 & 21 & 22 & 111 & 112 & 121 & \ldots
\end{array}
\]
in which the number $n$ corresponds to
\begin{itemize}
\item the $n$-th binary string $x$in lexicographic order
\item the string $x$ obtained by stripping the leading 1 from $(n+1)_2$,
the binary representation of $n+1$
\item the string $y=n_{\{1,2\}}$, the base 2 positional system using digits $\{1,2\}$.
\end{itemize}

Indeed, $n+1=2^l+\sum_{i=0}^{l-1}x_i 2^i$ iff $n=\sum_{i=0}^{l-1}2^i + \sum_{i=0}^{l-1}x_i 2^i$
iff $n=\sum_{i=0}^{l-1} (x_i+1) 2^i$.

\subsection{Prefix codes}

Another way to tie the natural numbers and binary strings together
is by what we shall call the {\em binary natural tree} shown in Figure~\ref{natree}.
It has the set of natural numbers as vertices,
and the set of binary strings as edges,
such that the $2^n$ length-$n$ strings
are the edges leading from vertex $n$.
Edge $w$ leads from vertex $|w|$ to $w+1$, which in binary is $1w$.

In~\cite{L68}, Vladimir Levenshtein defines a universal code for
the natural numbers that corresponds to
concatenating the edges on the path from $0$ to $n$, % which we'll denote by $p(n)$,
prefixed with a unary encoding of the depth of vertex $n$ in the tree.
%The importance of the binary natural tree lies in the observation that
The resulting set of codewords is {\em prefix-free}.
meaning that no string is a proper prefix of another,
which is the same as saying that the strings in the set are self-delimiting.
Prefix-free sets satisfy the {\em Kraft inequality}: $\sum_{s} 2^{-|s|} \leq 1$.
We've already seen two important examples of prefix-free sets, namely
the set of $\lambda$ term encodings $\widehat{M}$ and the
set of combinator encodings $\widetilde{M}$.
%For $p(m)$ to be a proper prefix of $p(n)$ requires
%$m$ to be an ancestor of $n$, which implies that $n$ is exponentially
%bigger than $m$.
%To turn $p(n)$ into a prefix code, it suffices
%to prepend the depth of vertex $n$ in the tree, i.e.
%the number of times we have to map $n$ to $|n-1|$ before we get to $\epsilon$.
The Levenshtein code $\brr{n}$ of a number $n$ can be defined recursively as
%\[\brr{n} = 1^{l^{\ast}(n)}0\ p(n), \] or, equivalently,
\[\brr{0} = 0\ \ \ \ \ \ \ \ \brr{n+1} = 1\ \brr{l(n)}\ n. \]

and satisfies the following nice properties:

\begin{itemize}
\item prefix-free and complete: $\sum_{n \geq 0} 2^{-|\brr{n}|} = 1$.
\item identifies lexicographic with numeric order: $\brr{m}$ lexicographically precedes $\brr{n}$ if and only if $m<n$.
\item simple to encode and decode
\item efficient in that for every $k$:
$|\brr{n}| \leq l(n)+l(l(n))+\cdots+ l^{k-1}(n) + O(l^k(n))$,
where $l(s)$ denotes the length of a string $s$.
\end{itemize}

\begin{figure}
\epsfxsize=8cm \epsfbox{natree.eps}
\caption{binary natural tree}
\label{natree}
\end{figure}

\begin{figure}
\epsfxsize=13cm \epsfbox{kraft.eps}
\caption{codes on the unit interval;
$\brr{0} = 0, \brr{1} = 10, \brr{2} = 110\ 0, \brr{3} = 110\ 1, \brr{4} = 1110\ 0\ 00, \brr{5} = 1110\ 0\ 01, \brr{6} = 1110\ 0\ 10, \brr{7} = 1110\ 0\ 11, \brr{8} = 1110\ 1\ 000, etc.$.}
\label{kraft}
\end{figure}

Figure~\ref{kraft} shows the codes as segments of the unit interval,
where code $x$ covers all the real numbers whose binary expansion starts
as $0.x$, and lexicographic order translates into left-to-right order.
%With a width of only $2^{-12}$, the code for 16 is too narrow to see, so we just
%indicate its location along with some later codes. 

%Compared to unary, this code is two bits longer on 4, one bit longer on 2 and
%5, equally long on 0,1,3 and 6, and becomes exponentially shorter in the limit.

\section{Upper bounds on complexity}
Having provided concrete definitions of all key ingredients of algorithmic
information theory, it is time to prove some concrete results about
the complexity of strings.

The simple complexity of a string is upper bounded by its length:
\[ KS(x) \leq |\widehat{\CI}|+l(x) = l(x)+4 \]

%The prefix complexity of a string given its length is also
%upper bounded by its length:
%\[ KP(x|l(x)) \leq |\widehat{{\bf readbits}}|+l(x) = l(x)+235, \]
The prefix complexity of a string is
upper bounded by the length of its delimited version:
\[ KP(x) \leq |\widehat{{\bf delimit}}|+l(\brr{x}) = l(\brr{x})+338. \]

where {\bf delimit} is an optimized translation
of the following code
{\small
\begin{verbatim}
let
 id = \x x;
 nil = \x\y y;
 -- readbit cont church_(n+1) list returns cont church_(n_list+1)
 -- where n_list is the number corresponding to the concatenation of n and list
 readbit = \cont\n\list list (\bit cont (\f\x (n f (n f (bit x (f x))))));
 -- dlmt cont list returns cont church_n rest_list
 dlmt = \cont\list list (\bit bit (cont nil) (dlmt (\len len readbit cont id)));
 -- incc cont done list return either cont (if carry) or done (if no carry) of incremented list
 incc = \cont\done\list list (\msb incc (\r\_ msb done cont (\z z (\x\y msb y x) r))
                                        (\r\_     done      (\z z       msb      r))
                             ) (cont list);
 -- inc list returns incremented list
 inc = incc (\r\z z (\x\y x) r) id;
in dlmt (\n\l\z z (n inc nil) l)
\end{verbatim}
}

The prefix complexity of a pair is upper bounded by the sum of
individual prefix complexities, one of which is conditional on
the shortest program of the other:
\[  K(x,y) \leq K(x) + K(y|x^{\ast}) + 657. \]

This is the easy side of the fundamental ``Symmetry of information''
theorem $K(x)-K(x|y^{\ast}) = K(y)-K(y|x^{\ast}) + O(1)$, which
says that $y$ contains as much information about $x$ as $x$ does about $y$.

In~\cite{ch01}, Chaitin proves the same theorem using a resource bounded evaluator,
which in his version of LISP comes as a primitive called "try".
His proof is embodied in the program gamma:

{\small
\begin{verbatim}
((' (lambda (loop) ((' (lambda (x*) ((' (lambda (x) ((' (lambda (y) (cons x
(cons y nil)))) (eval (cons (' (read-exp)) (cons (cons ' (cons x* nil))
nil)))))) (car (cdr (try no-time-limit (' (eval (read-exp))) x*)))))) (loop
nil)))) (' (lambda (p) (if(= success (car (try no-time-limit (' (eval
(read-exp))) p))) p (loop (append p (cons (read-bit) nil)))))))
\end{verbatim}
}

of length 2872 bits.

We constructed an equivalent of "try" from scratch.
The constant 657 is the size of the term {\tt pairup} defined below,
containing a monadic lambda calculus interpreter that tracks the input bits read so far
(which due to space restrictions is only sparsely commented):

{\small
\begin{verbatim}
let
  id = \x x;
  true = \x \y x;
  false = \x \y y;
  nil = false;
  -- monadic parser
  uni = \abs \app let uni0 = \cnt \ps \xs
    xs (\b0 let ps0 = \ts ps (\p p b0 ts) in
      \ys \uni0 \cnt ys (\b1
        let ps1 = \ts ps0 (\p p b1 ts) in
          b0 (uni0 (\v1 (b1 (cnt (\ctx abs (\v2 v1 (\p p v2 ctx))))
                           (uni0 (\v2 cnt (\ctx app (v1 ctx) (v2 ctx)))))))
             (b1 (cnt (\ctx ctx b1))
                 (\d \d uni0 (\v cnt (\ctx v (ctx b1))) ps0 ys)) ps1)) uni0 cnt
  in uni0;
  -- this monad pairs up lambdas with some attribute
  abs = \v \p v; app = \ca ca false;
  -- pair each suffix with a prefix attribute
  list = \x \xs \ys
     let mx = abs (\e abs (x e));
         mxs = xs list (\rs ys (\p p x rs)) in
     \p p ys (\e e false mx false mxs);
  pairup = uni abs app (\prog \ps \pi
    let pspi = \d (
          prog d             -- provide ctx
          d                  -- monadic apply
          (pi list ps)       -- to monadic input, expect monadic <poM,qsM>
          false              -- get <poM,qsM>; can't use d here in case prog is id
          (\p \e \p \e e) )  -- monadic tail
          true               -- head of qsM is difflist of bits read
          nil);              -- terminate list
        -- Direct evaluator, using the identity monad.
        eval = uni id id id pi
    in eval (ps pi) (\po \qs eval qs (pspi qs) (\qo \rs \p p (\p p po qo) rs))) id;
in pairup
\end{verbatim}
}

Although more involved, our program is less than a quarter the size of Chaitin's
when measured in bits. Chaitin also offered a program of size 2104 bits,
at the cost of introducing yet another primitive into his language,
which is still 220\% longer than ours.

%%\[ KP(x,y) \leq |\widehat{M}| + KP(x) + KP(y) = KP(x) + KP(y) +  ,\]
%%where $M \equiv (\lambda u\ . )U$ corresponds to the Haskell program
%[  Proof that H(x,y) <= H(x) + H(y|x) + 2872.  ]
%[  The 2872-bit prefix gamma proves this.      ]
%   [read program p bit by bit until we get it all]
%   let (loop p)
%      if = success car try no-time-limit 'eval read-exp p
%      [then] p
%      [else] (loop append p cons read-bit nil)
%%\begin{verbatim}
%%pairprogram = do x <- U
%%                 y <- U
%%                 return (x,y)
%%\end{verbatim}

\section{Future Research}
It would be nice to have an objective measure of the simplicity and expressiveness
of a universal machine. Sizes of constants in fundamental theorems are an indication,
but one that is all too easily abused. Perhaps diophantine equations can serve as
a non-arbitrary language into which to express the computations underlying a proposed
definition of algorithmic complexity, as Chaitin has demonstrated for relating the existence
of infinitely many solutions to the random halting probability $\Omega$.
Speaking of $\Omega$, our model provides a well-defined notion of halting as well,
namely when $\CU\ \pref{p}{z} = \tup{M,z}$ for any term $M$ (we might as well allow $M$ without
normal form). Computing upper and lower bounds on the value of $\Omega_{\lambda}$,
as Chaitin did for his LISP-based $\Omega$, and Calude et al. for various other languages,
should be of interest as well.
A big task remains in finding a good constant for the other direction of
the `Symmetry of Information' theorem, for which Chaitin has sketched a program.
That constant is bigger by an order of magnitude, making its optimization
an everlasting challenge.

\section{Conclusion}
The $\lambda$-calculus is a surprisingly versatile and concise language,
in which not only standard programming constructs like bits, tests, recursion,
pairs and lists, but also reflection, reification, and marshalling are
readily defined, offering an elegant concrete foundation
of algorithmic information theory.
%Obtaining stronger results such as $KP(y|x^\ast) = KP(x,y) − KP(x) + O(1)$
%requires a bit more work such as implementing a 

An implementation of Lambda Calculus, Combinatory Logic,
along with their binary and universal versions,
written in Haskell, is available at~\cite{tromp}.

\section{Acknowledgements}
I am greatly indebted to Paul Vit\'{a}nyi for fostering my research into concrete definitions
of Kolmogorov complexity, and to Robert Solovay, Christopher Hendrie and Bertram Felgenhauer
for illuminating discussions on my definitions and improvements in program sizes.
\begin{thebibliography}{9}

\bibitem{pen}
R. Penrose, {\it The Emperor's New Mind},
Oxford University press, 1989.

\bibitem{ch}
G. Chaitin, {\it An Invitation to Algorithmic Information Theory},
DMTCS'96 Proceedings, Springer Verlag, Singapore, 1997, pp. 1--23
(http://www.cs.auckland.ac.nz/CDMTCS/chaitin/inv.html).

\bibitem{ch01}
G. Chaitin, {\it Exploring Randomness}, Springer Verlag, 2001.
(http://www.cs.auckland.ac.nz/CDMTCS/chaitin/ait3.html)

\bibitem{mul}
R. Muller, {\it M-LISP: A representation-independent dialect of LISP
with reduction semantics}, ACM Transactions on Programming Languages
and Systems 14(4), 589--616, 1992.

\bibitem{lev}
L. Levin,
{\it On a Concrete Method of Assigning Complexity Measures},
Doklady Akademii nauk SSSR, vol. 18(3), pp. 727--731, 1977. 

\bibitem{lv}
M. Li and P. Vit\'anyi,
{\it An Introduction to Kolmogorov Complexity and Its Applications},
Graduate Texts in Computer Science, second edition, Springer-Verlag,
New York, 1997.

\bibitem{kl}
S.C. Kleene,
{\it Lambda-Definability and Recursiveness},
Duke Mathematical Journal, 2, 340--353, 1936.

\bibitem{L68}
Vladimir Levenshtein,
{\it On the redundancy and delay of decodable coding of natural numbers},
Systems Theory Research, 20, 149--155, 1968.

\bibitem{PE88}
Frank Pfenning and Conal Elliot,
{\it Higher-Order Abstract Syntax},
ACM SIGPLAN'88 Conference on Programming Language Design and Implementation,
199--208, 1988.

\bibitem{fw}
D. Friedman and M. Wand,
{\it Reification: Reflection without Metaphysics},
Proc. ACM Symposium on
LISP and Functional Programming, 348--355, 1984.

\bibitem{fs}
Gudmund S. Frandsen and Carl Sturtivant,
{\it What is an Efficient Implementation of the $\lambda$-calculus?},
Proc. ACM Conference on Functional Programming and Computer Architecture
(J. Hughes, ed.), LNCS 523, 289--312, 1991.

\bibitem{SM89}
J. Steensgaard-Madsen,
{\it Typed representation of objects by functions},
TOPLAS 11-1, 67--89, 1989.

\bibitem{bruijn}
N.G. de Bruijn,
{\it Lambda calculus notation with nameless dummies, a tool for automatic
formula manipulation},
Indagationes Mathematicae 34, 381--392, 1972.

\bibitem{baay}
H.P. Barendregt,
{\it Discriminating coded lambda terms},
in (A. Anderson and M. Zeleny eds.) Logic, Meaning and Computation, Kluwer,
275--285, 2001.

\bibitem{bertram}
Bertram Felgenhauer, private communication, aug 27, 2011.

\bibitem{dm}
François-Nicola Demers and Jacques Malenfant,
{\it Reflection in logic, functional and object-oriented programming: a Short
Comparative Study},
Proc. IJCAI Workshop on Reflection and Metalevel Architectures
and their Applications in AI, 29--38, 1995.

\bibitem{fwh}
Daniel P. Friedman, Mitchell Wand, and Christopher T. Haynes,
{\it Essentials of Programming Languages -- 2nd ed},
MIT Press, 2001.

\bibitem{mc}
Oleg Mazonka and Daniel B. Cristofani,
{\it A Very Short Self-Interpreter},
\verb%http://arxiv.org/html/cs.PL/0311032%,
2003.

\bibitem{geb}
D. Hofstadter, {\it Godel, Escher, Bach: an Eternal Golden Braid},
Basic Books, Inc., 1979.

\bibitem{b}
H.P. Barendregt,
{\it The Lambda Calculus, its Syntax and Semantics},
%Studies in Logic and The Foundations of Mathematics vol. 103,
revised edition,
North-Holland, Amsterdam, 1984.

\bibitem{KKSV97}
J.R. Kennaway and J. W. Klop and M.R. Sleep and F.J. de Vries,
{\it Infinitary Lambda Calculus},
Theoretical Compututer Science, 175(1), 93--125,1997.

\bibitem{spj}
Simon Peyton Jones,
{\it  Tackling the awkward squad: monadic input/output, concurrency, exceptions, and foreign-language calls in Haskell},
in "Engineering theories of software construction", ed. Tony Hoare,
Manfred Broy, Ralf Steinbruggen, IOS Press, 47--96, 2001.

\bibitem{hk}
The Haskell Home Page, \verb%http://haskell.org/%.

\bibitem{bf}
Brainfuck homepage, \verb%http://www.muppetlabs.com/~breadbox/bf/%.

\bibitem{m00}
Torben \AE. Mogensen,
{\it Linear-Time Self-Interpretation of the Pure Lambda Calculus},
Higher-Order and Symbolic Computation 13(3), 217-237, 2000.

\bibitem{t}
D. A. Turner, {\it Another algorithm for bracket abstraction},
J. Symbol. Logic 44(2), 267--270, 1979.

\bibitem{tromp}
J. T. Tromp,
\verb%http://www.cwi.nl/~tromp/cl/cl.html%, 2004.

\end{thebibliography}
\end{document}
